\documentclass{article}

\usepackage[a4paper, margin=1in]{geometry}

\usepackage{graphicx}
\graphicspath{ {./images/} }

\title{Informação Profissional em Ciência da Computação:\\
	Arquitetura de Computadores}
\author{Carlos Eduardo Gallo Filho \\
	Caio \\
	Pedro}

\date{\today}

\begin{document}

\maketitle

\section{Introdução}
\subsection{Arquitetura e Organização}
\subsection{Estruturas e Funções}
\subsubsection{Múltiplos núcleos}
\subsubsection{Estrutura interna do núcleo}

\section{Evolução histórica}
\subsection{Primeira geração: Arquitetura de Von Neumann}
A primeira geração de computadores é conhecida pelo uso das válvulas para
representar os elementos lógicos digitais e a memória. E também, o
\textit{conceito de programa armazenado}, atribuída ao matemático John Von
Neumann, que surgiu para a construção do computador EDVAC (Eletronic Discrete
Variable Computer), mas foi principalmente discutido no desenvolvimento do
computador IAS, o qual é um protótipo para quase todos os computadores de
propósito geral de hoje em dia.

\begin{figure}[h]
    \caption{Estrutura do IAS.}
    \centering
    \includegraphics[width=0.75\textwidth]{ias.png}
\end{figure}

Para se explicar a estrutura do IAS, deve-se atentar a 5 partes principais:

\begin{enumerate}
    \item Um computador terá de ser capaz de executar as
	operações elementares básicas (adição, subtração, multiplicação,
	divisão). Normalmente, para essas funções são criadas unidades
	específicas para tais, comportadas em uma unidade maior
	centralizadora denominada CA ou unidade lógica e aritmética.

    \item Um computador terá de ser capaz de dar sequenciamento adequado as
	suas operações, instruções e as instruções de controle (comandos que
	regem o próprio sequenciamento do computador). Á essas undides é
	denomiada uma unidade central chamada CC ou controle central.

	As partes 1) e 2) juntas são chamadas de C. 

    \item Um computador terá a necessidade de processar sequências longas de
	operações, que, geralmente, não conseguem ser efetuadas em uma única
	leva de comandos. Logo, requer-se uma unidade que armazene dados por
	períodos duráveis para resolver problemas mais complexos, como
	cálculos.

	Essa unidade é denominada M ou memória.

	Analogamente ao funcionamento do corpo humano, essas três partes podem
	ser comparadas a parte cognitiva do corpo. Ou seja, a unidade lógica e
	aritmética, o controle central e a memória são o conjunto pensante do
	computador. Mas, assim como o corpo humano, se faz necessário a
	comunicação com o mundo externo, tal é feito por um canal denominado
	meio de gravação de sáida do dispositivo ou R, de modo que as últimas
	duas partes interagem com esse canal para satisfazer a necessidade de
	interação com o externo. 

    \item A parte de entrada do meio R, a qual deve transferir as informações
	desse para dentro de M e C é chamada de entrada ou I. É de boa prática a
	entrada passar os dados para dentro de M e nunca diretamenta para C. 

    \item Similarmente, a parte de saída do meio R, a qual deve transferir as
	informações de M e C para o meio R, é chamada de saída ou O. É de boa
	prática, também, a saída passar de M para R, assim, nunca diretamente
	por C. 
\end{enumerate}

Assim, resumidamente, temos a memória principal, que armazena dados e
instruções; a unidade lógica e aritmética (ALU), que opera os binários; a
unidade de controle, que interpreta e executa instruções e, por fim, o
equipmaneto de saída (E/S), que conecta o meio interno ao meio externo do
computador.

\subsubsection{Endereçamento de memória}

\subsection{Segunda geração: Transístores}
\subsection{Terceira geração: Circuitos Integrados}
\subsection{Gerações posteriores}

\section{Aplicações} 
As subseções a seguir trazem algumas aplicações da arquitetura de computadores.
\subsection{Paradigmas de arquitetura}

\subsubsection{CISC}
O paradigma de arquitetura CISC (\textit{Complex Instruction Set Computer})
denotam uma escolha de projeto na qual o conjunto das instruções utilizadas são
projetadas para a execução de um conjunto de operações de baixo nível,
envolvendo por exemplo, operações aritméticas juntas de uma escrita na memória,
tudo em uma mesma instrução.

Uma notória vantagem do uso do paradigma CISC se dá por facilitar a construção
de compiladores e a programação em baixo nível, pois as instruções fornecem um
pequeno nível de abstração. Além disso, por implementar as instruções a comuns
nível de \texti{hardware}, o consumo de memória, tempo de execução e tamanho
dos programas costumam ser menores. Uma desvantagem direta ao paradigma é o
fato do projeto do hardware se tornar muito mais trabalhoso e complexo, além de
necessitar maior quantidade de componentes, e por consequência, aumento de
calor.

\subsection{RISC}
Em contrapartida ao CISC, existe o paragima RISC (\textit{Reduced Instruction
Set Computer}). Um projeto RISC tem como princípio implementar um pequeno
número mínimo de instruções simples, que em sua maioria são executadas em
apenas um cíclo de máquina.

Como esperado, processadores RISCs são mais fáceis de serem implementados e
produzidos, gastando menor quantidade de componentes e melhorando a eficiência
térmica, além de facilitar a implementação do \textit{pipelining} a nível do
código de máquina. Como desvantagem, sua programação de baixo nível é mais
complexa, bem como a construção de compiladores, o que implica uma má
implementação a nível de software ser mais danosa.

\subsection{Arquitetura x86} 
A arquitetura de processadores x86 foi concebida pela intel e é o exemplar mais
popular do tipo CISC. 

\subsection{Arquitetura ARM} 
\subsection{Computação em Nuvem}

\end{document}
